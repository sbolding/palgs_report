% --------------------------------------------------------------
% This is all preamble stuff that you don't have to worry about.
% Head down to where it says "Start here"
% --------------------------------------------------------------
 
\documentclass[12pt]{article}

\usepackage{courier}
\usepackage{color}
\usepackage{listings}
\usepackage[square,numbers]{natbib}
\usepackage{tabls}
\usepackage{graphicx}
\usepackage{subcaption}
\usepackage{pdfpages}
\usepackage{mathtools}
\usepackage{enumitem}
\usepackage{hyperref}
\usepackage{multirow}
\definecolor{dkgreen}{rgb}{0,0.6,0}
\definecolor{gray}{rgb}{0.5,0.5,0.5}




\lstset{language=Matlab,
   keywords={break,case,catch,continue,else,elseif,end,for,function,
      global,if,otherwise,persistent,return,switch,try,while},
   basicstyle=\ttfamily,
   keywordstyle=\color{blue},
   commentstyle=\color{red},
   stringstyle=\color{dkgreen},
   numbers=left,
   numberstyle=\tiny\color{gray},
   stepnumber=1,
   numbersep=10pt,
   backgroundcolor=\color{white},
   tabsize=4,
   showspaces=false,
   showstringspaces=false}
 
\usepackage[margin=1in]{geometry} 
\usepackage{amsmath,amsthm,amssymb}
\usepackage{verbatim}
\usepackage{algpseudocode,algorithm}
\usepackage{setspace}

\newcommand{\lline}{\noindent\makebox[\linewidth]{\rule{\textwidth}{0.4pt}}}
\newcommand{\N}{\mathbb{N}}
\newcommand{\Z}{\mathbb{Z}}
\newcommand{\deriv}[2]{\frac{\mathrm{d} #1}{\mathrm{d} #2}}
\newcommand{\pderiv}[2]{\frac{\partial #1}{\partial #2}}
\newcommand{\bx}{\mathbf{X}}
\newcommand{\ba}{\mathbf{A}}
\renewcommand{\d}{\mathrm{d}}
\newcommand{\upl}{u_{\text{plane}}}
\newcommand{\upt}{u_{\text{point}}}
\newcommand{\D}{\Delta}
\renewcommand{\SS}{\State}
 
\newenvironment{theorem}[2][Theorem]{\begin{trivlist}
\item[\hskip \labelsep {\bfseries #1}\hskip \labelsep {\bfseries #2.}]}{\end{trivlist}}
\newenvironment{lemma}[2][Lemma]{\begin{trivlist}
\item[\hskip \labelsep {\bfseries #1}\hskip \labelsep {\bfseries #2.}]}{\end{trivlist}}
\newenvironment{exercise}[2][Exercise]{\begin{trivlist}
\item[\hskip \labelsep {\bfseries #1}\hskip \labelsep {\bfseries #2.}]}{\end{trivlist}}
\newenvironment{problem}[2][Problem]{\begin{trivlist}
\item[\hskip \labelsep {\bfseries #1}\hskip \labelsep {\bfseries #2:}]\hspace{0.3in}\newline\newline}{\end{trivlist}}
\newenvironment{question}[2][Question]{\begin{trivlist}
\item[\hskip \labelsep {\bfseries #1}\hskip \labelsep {\bfseries #2.}]}{\end{trivlist}}
\newenvironment{corollary}[2][Corollary]{\begin{trivlist}
\item[\hskip \labelsep {\bfseries #1}\hskip \labelsep {\bfseries #2.} ]}{\end{trivlist}}
\newenvironment{problem*}[1][Problem]{\begin{trivlist}
\item[\hskip \labelsep {\bfseries #1} {\hspace{-0.2em}\bfseries:}]}{\end{trivlist}}
\newenvironment{solution}[1][Solution]{\begin{trivlist}
\item[\hskip \labelsep {\bfseries #1} {\hspace{-0.2em}\bfseries:}]\hspace{0.3in}\newline}{\end{trivlist}}
 
\begin{document}
 
% --------------------------------------------------------------
%                         Start here
% --------------------------------------------------------------
 
\title{Programming Assignment 2}%replace X with the appropriate number
\author{Simon Bolding\\ %replace with your name
CSCE 626} %if necessary, replace with your course title
 
\maketitle

\clearpage

%\includepdf[pages={1}]{p1p3.pdf}

\section*{References}

\begin{enumerate}
	\item People in class - Hans Hammer, Daniel Holladay, Josh Hansel
	\item wikipedia.org/wiki/Prefix\_sum,computing.llnl.gov, mpitutorial.com
    \item Class notes
    \item EOS website: sc.tamu.edu/systems/eos
\end{enumerate}

\section{Introduction}

In this work experimental analysis was performed to evaluate the performance
of two different methods for computing prefix sums in parallel: an
algorithm implemented with MPI and an algorithm implemented using OpenMP.  Given a sequence of numbers $x_1, x_2, ..., x_n$, the prefix sums are the partial sums 
\begin{align*}
s_1 =&x_1  \\
s_2 =& x_1 + x_2  \\
\vdots& \\
s_n =& x_1 + x_2 + \cdots + x_n .
\end{align*}
Multiple experiments were performed, and the results were analyzed to compare the performance of the algorithms 
as a function of the number of parallel processors used, for both algorithms.
Coefficients for the theoretical asymptotic complexity were determined. Both strong
and weak scaling studies were performed as well, and speed up studied as a function
of the number of elements. The
algorithms are benchmarked against the best sequential algorithm.

\section{Theoretical Analysis}

\subsection{Sequential Algorithm}

An efficient, time optimal sequential algorithm was used as a reference to
compare to parallel algorithms.  The algorithm is given in Alg.~\ref{alg:1}.
This algorithm has asymptotic time complexity $T(n)=O(n)$, where $n$ is the number of
input integers. Because $n-1$ additions must be performed to calculate the prefix sum,
this is clearly the optimal sequential algorithm. It is noted as an implementation detail that the call to
    \verb{PrefixSumSerial{ uses the same function call in the parallel versions of
        the code.  This ensures there are no optimization differences for this
        function call.
\begin{algorithm}
\begin{algorithmic}
    \caption{Serial prefix sum algorithm\label{alg:1}}
\setstretch{1.1}
\Procedure {PrefixSumSerial}{$A,B$} \Comment{Assumes len($A$)=len($B$)=$n$}
    \State variables: $i$, $n$
    \State $n$ = len($A$)
    \State $B(0)$ := $A(0)$ \Comment{Initialize prefix sum}
    \For{$i=1$ to $n-1$}
        \SS B(i) := A(i)+B(i-1)
    \EndFor
    \Comment{$B$ now contains the prefix sum of $A$}
\EndProcedure
\end{algorithmic}
\end{algorithm}

\subsection{MPI Algorithm}

The algorithm for the MPI implementation is given in Alg.~\ref{alg:mpi}. This
algorithm uses the standard approach from the previous homework of computing local
prefix sums, followed by a global prefix sum on the last element in each local array,
before adding the sum of previous processors to each element. 

The algorithm assumes that the input elements and output prefix elements are
distributed across $p$ processors.  Thus, for $p$ processors and $n$ integers, each
processor contains $m\approx n/p$ elements.  If $n/p$ is not an integer, then the remaining
elements are distributed, one per processor, amongst the first $n\mod p$ processors.  Each processor
computes the prefix sum of its $m$ elements, an order $O(n/p)$ operation.  Using \verb{MPI_Allgather{, a copy of the last
    element in each processor's prefix sum array (representing the sum of all
    elements on each processor) is copied to \emph{all} processors.  Thus, each processor
    now contains an array $C$ of size $p$, with the $i$-th element containing the sum of the
    elements on the $i$-th processor.  The prefix sum of $C$ is computed by each
    processor, an $O(p)$ operation.  Finally, each processor adds the sum of integers from all previous processors to
    its local prefix sum array as necessary.
    
    For time and work complexity, the initial prefix sums and final value additions
    require $O(n/p)$ time and $O(n)$ work. To perform the global prefix sum, the
\verb{MPI_Allgather{ command was utilized to copy the processor sums to each
    processor. Although $p$ numbers must be communicated, by combining messages in a
    tree-reduction, this command only requires the large communication overhead of
    $O(\log p)$ messages. Thus, as far as asymptotic complexity, this process
    represents an $O(\log p)$ communication time, with $p \log p$ work.  The serial prefix sum of each
    processors array is computed only up to that processor, but this is bounded from
    above by $O(p)$ time and $O(p^2)$ work.  Thus, in total,
    $T(n)=O(n/p+p+log(p))\simeq O(n/p+p)$
    and $W(n)=O(n+p^2+p\log p) \simeq O(n+p^2)$.  It is noted that for the case of the
    experiments performed in this work,
    $n\gg p$, and the $O(p)$ terms become negligible.
    We have verified this
algorithm works in previous homework.


\begin{algorithm}
\begin{algorithmic}
    \caption{MPI prefix sum algorithm\label{alg:mpi}}
\setstretch{1.1}
\Procedure {PrefixSumMPI}{$A$} \Comment{Assumes len($A$)=$n$}
    \State \Comment{Assume $A$ and $B$ distributed evenly on processors, $\approx n/p$ to each processor}
    \State Local variables: $id$, $p$, $m$ \Comment{MPI Rank, number of processors,
    my size}
    \State Local variables: $C,D$ \Comment{Arrays of size $p$ for parallel prefix sum}
    \State $id$ := get\_my\_id() \Comment{Procs. numbered $0,1,\ldots,n-1$}
    \State $m$ := $\lfloor n/p \rfloor$ \Comment{Determine size on each processor}
    \If{$id < n \mod p$}
    \State $m$ :=  $m+1$
    \EndIf
    \State Local variables: $myints$ \Comment{this procs' $m$ input values of $A$}
    \State Local variables: $mypsums$ \Comment{this procs' $m$ prefix sum values of $B$}
    \If{ len($A$) = 1}
        \SS $B(0)$ := $A(0)$
        \SS \textbf{return}
    \Else
        \SS PrefixSumSerial($myints$,$mypsums$)
    \EndIf
    \State MPI\_Allgather($mypsums(m-1)$,$C$) \Comment{$C$ now contains copy of sum
        of elements for \emph{each} of $p$ processors}
    \If{$id>0$}
    $D(0) := C(0)$
    \For{$i=0$ to $id-1$} \Comment{All Procs perform prefix sum on their local copy of $C$}
        \State $D(i)$ := $D(i-1)$+$C(i)$
    \EndFor
        \For{$i=0$ to $m$}
            \SS $mypsums(i)$ := $mypsums(i)$+$D(id-1)$
        \EndFor
    \EndIf \Comment{Collectively the $p$ arrays $mypsums$ contain prefix sum of $A$}
\EndProcedure
\end{algorithmic}
\end{algorithm}

\subsection{OpenMP Algorithm}

The OpenMP algorithm is given in Alg.~\ref{alg:openmp}.  The OpenMP algorithm works similarly to the MPI algorithm above.  Each thread performs the
local prefix sum on its portion of the input array, before a serial prefix sum is
performed on the $p$ processor sums, stored in an additional array.  This step is performed
with the simple serial algorithm by one processor.  This is done because for the
shared memory experiments performed, $p\ll n$, and the memory overhead of using a more
complicated algorithm would ultimately result in a less efficient algorithm and
longer run times.  Finally, the sum of previous processor's values are added
to each portion of the array in parallel, as necessary.  

For time complexity, the initial prefix sum and addition of values at the end are
$O(n/p)$, with work complexity $O(n)$.  The prefix sum of the processor sums is
$O(p)$ time, but, because there is shared memory access, this step can be performed
by a single processor.  This results in $O(p)$ work.  In total, for this algorithm,
$T(n)=O(n/p+p)$, where $p\ll n$, and Work is $O(n+p)$. We have verified this
algorithm works in previous homework.

\begin{algorithm}[t!]
\begin{algorithmic}
    \caption{OpenMP prefix sum algorithm\label{alg:openmp}}
\setstretch{1.1}
\Procedure {PrefixSumOpenMP}{$A$} \Comment{Assumes len($A$)=$n$}
    \State \Comment{Assume $A$ and $B$ are in shared memory among $n$ processors}
    \State Local variables: $id$, $p$, $m$ \Comment{MPI Rank, number of processors,
    my size}
    \State Local variables: $C,D$ \Comment{Arrays of size $p$ for parallel prefix sum}
    \State $id$ := get\_my\_id() \Comment{Threads numbered $0,1,\ldots,n-1$}
    \State Local variable: $mybeg$, $myend$ \Comment{Determine my portion of the array}
    \State $m$ := $\lfloor n/p \rfloor$ \Comment{Initial size for each processor}
    \State $r$ := $n \mod p$
    \If{$id < r$} \Comment{Account for $n/p$ not an integer}
        \State $mybeg$ := $(m+1)*id$
        \State $myend$ := $mybeg\,+\,m$ 
    \Else
        \State $mybeg$ := $r*(m+1)+(id-r)*m$
        \State $myend$ := $mybeg+(m-1)$
    \EndIf
    \If{ len($A$) = 1}
        \SS $B(0)$ := $A(0)$
        \SS \textbf{return}
    \Else
        \SS PrefixSumSerial($A(mybeg:myend)$, $B(mybeg:myend)$) \Comment{$myend$ is
            inclusive}
    \EndIf
    \SS $C(id)$ := $B(myend)$ \Comment{Store sum of each thread's ints}
    \If {$id$ = 0}
    \SS PrefixSumSerial($C$,$D$) \Comment{$D$ contains prefix sums of each thread}
    \Else
     \For{$i=mybegin$ to $myend$}
            \SS $B(i)$ := $B(i)$+$D(id-1)$
        \EndFor
    \EndIf \Comment{$B$ now contains the prefix sum of $A$}
\EndProcedure
\end{algorithmic}
\end{algorithm}

\section{Experimental Setup}

\subsection{Machine Information}

The parallel programs were tested on \emph{eos}, a machine at Texas A\&M. For all the
results in this work, the available Intel ``Nehalem'' nodes were used. These
processors use Intel 64-bit architecture.  Each node contains two sockets, each with
a chip containing 4 processing units, resulting in 8 processing units per core.  There is some potential difference in
memory access times on the chip when going from 4 to 8 cores, where memory must be
accessed off chip.  The interconnection of nodes is done using a ``Fat Tree'' topology.  This
results in a constant communication time to access any off board node from any other.

For memory, each core has 32 kB L1 cache and 256 kB of unified L2 cache. Each Nehalem
chip (containing four cores) has an 8 MB shared L3 cache.  There is $\sim 22$ GB of
shared RAM available to each node (i.e., 2 chips, or 8 cores). The RAM has non-uniform access time,
with longer access times when a core accesses the DRAM that is located near the other
chip on that node. More details about the architecture of \emph{eos} can be found
at \url{http://sc.tamu.edu/systems/eos/}

\subsection{Description of Experiments}

Several experiments were performed to gauge the performance and scalability of Alg.~\ref{alg:mpi}
and~\ref{alg:openmp}. The experiments are discussed individually below. Batch files to run the
various jobs were created using a Python script.  Output files were processed with a
Python script as well.  Example scripts can be found in the submitted tar ball.  It
is noted that the measured simulation times for the experiments times only account for the prefix sum
computations; they exclude any extra input, output, or initialization timing costs.

\subsubsection{Strong Scaling Study} 

The purpose of the strong scaling
study is to see how much faster a problem of a fixed size can be solved by
using more processors.  Speed up was used as a performance measure for the strong
scaling study.   Speed up is defined as the ratio $T_{ser}/T_p$, where $T_{ser}$ is the time to solve the problem using the most
efficient serial algorithm and $T_{p}$ is the time to solve the program using $p$
processors. This is different than scalability,
which is the ratio $T_{1}/T_{p}$, which can also be used as a performance measure for
strong scaling.  A program which scales perfectly would show a linear, one-to-one
speed up.  In general this is not the case due to communication and memory overhead.

For both the MPI and OpenMP algorithm, $10^9$ integers was chosen as the input
problem size. The choice of this number was to ensure that the size of each portion of
the input and prefix sum arrays stored by each processor was larger than the $L3$
cache, for all simulations.  The size of the input array for $10^9$ integers is $4$ bytes/int
$\times 10^9$ integers $=4$ GB.  The size of the output arrays, which use 8 byte
longs, is $8\times 10^9$ GB. In total, around $12$ GB of memory is needed, which is well
under the limit of $22$ GB per node.  The largest run of 256 cores still requires
around $10$ MB per core, which is greater than the size of the L3 cache.


\subsubsection{Weak Scaling Study}

A weak scaling study determines the efficiency of the algorithm as you increase the
number of processors, while keeping the problem size \emph{per core} fixed.
The goal of a weak scaling study is to determine the increased cost of an algorithm
as more processors are used to solve increasingly larger problems. This indicates how
well an algorithm can be used to solve problems that may be too large to solve with a
serial algorithm, or even at lower core counts.  The metric for the weak scaling studies used was efficiency,
defined as Efficiency = $T_p(n)/T_1(n)\times 100\%$, noting that $T_1$ is the time for the parallel
algorithm with one processor, not the time for the serial algorithm.  The ideal
efficiency would be 100\%, resulting in a flat line for Efficiency as a function of
$p$.  Decreases in efficiency likely indicate cost increase from overhead due to memory access or
communication times.

For the weak scaling studies, $10^8$ integers, per core, were used for both the MPI
and OpenMP.  This number was chosen for the same reasons as in the strong scaling
study: the value is larger than the cache size and the total problem size will fit in
the available RAM per node.

\subsubsection{Speedup Versus Problem Size}

A study similar to strong scaling was performed, in which the problem size is varied,
but the number of processors is fixed.  Thus, the interest is in whether the
algorithm, for a particular $p$, behaves as expected when $n$ is increased.  This
helps to validate our asymptotic analysis, and will determine if the $O(n/p)$ term is
truly dominant as anticipated.  This study provides information similar to the strong and weak scaling studies, but
should expose any scaling issues that are not necessarily related to communication,
e.g., memory issues.
It also provides a good overall indication of the algorithms' ability to solve
problems of varying size more efficiently than the serial algorithm.

For a couple different choices of $p$, the algorithms were tested on various problem
sizes $n$. The run times of these simulations is then compared to the run times for
the same test problems by the serial program.  

\subsubsection{Determining Asymptotic Coefficients}

To determine the validity of our theoretical analysis of the algorithms, simulations
were performed for a fixed number of processors with variable input sizes.  This
helps to determine if our theoretical model for run time, as a function of input
size and number of processors, is sufficiently accurate to model the run time.  Our model does not account for
communication latencies or memory access times.  This study also helps to determine in what
range of problem sizes our asymptotic scalings are accurate.  

The theoretical output time of the model is predicted as $T_{pred}$. In general,
the model $T_{pred}$ is a function of the problem size $n$ and number of processors $p$.
The model can be represented as $T_{pred} = C_0 g(n,p)$ for some $n>n_0$, where
$g(n,p)$ is the expected complexity determined by algorithmic analysis, for a given
compute system.  By performing experiments for various $n$, given a fixed $p$, the
coefficients of the model $C_0$ and $n_0$ can be determined by plotting the ratio of
the actual experimental time $T_{exp}$ to predicted run times.  The point at which
the plot levels off represents $n_0$, and the value of the ratio where it has leveled
off approximates $C_0$.

For both the OpenMP and MPI algorithm, the dominant term is expected to be $O(n/p)$.
Rather than trying to fit a function to determine the other coefficients in the
model (e.g., $C_0*n/p+C_1*p + C_2*\log p$), which would be difficult due to statistical noise in the results and the small
contribution from the $O(p)$ terms, only the coefficients for the dominant $O(n/p)$
term are determined.  


\subsection{Statistics}

The experiments must be repeated to measure various forms of variability in the
system, e.g., variable communication time, memory access times, inaccuracy of the
timer, etc.  Unless noted otherwise, the experiments were repeated 32 times for each
plotted data point.  The entire program is rerun for each iteration to ensure the
effect of variability in memory initialization times on total execution time is represented
accurately.  

From all 32 repeated simulations, the reported run times are simply the average of
the particular result from 32
simulations. The standard error in the average of a quantity is $\sigma/\sqrt{N}$,
where $\sigma$ is the sample standard deviation of the quantity from all 32 simulations.  Since speed up and
scalability are calculated quantities with a statistical variance in both terms, it
is necessary to approximate the error in the quantity.  
Based on the standard error propagation formula
(\url{http://en.wikipedia.org/wiki/Propagation_of_uncertainty}), the error for the
ratio of two timing results $T_i$ and $T_j$ is
\begin{equation}
    \sigma \left( \frac{T_i}{T_j} \right) = \frac{T_i}{T_j}
    \sqrt{\left(\frac{\sigma_{T_i}}{T_i}^2\right) + 
    \left(\frac{\sigma_{T_j}}{T_j}^2\right)}
    \label{err}.
\end{equation}
The above equation is used to determine the standard error for all plotted speed ups
and scalability.  The plotted values are the 95\% confidence interval. This
confidence interval, assuming a
Gaussian distribution of the error, is plotted as $1.96 \sigma$.

\section{Experimental Results and Analysis.}

Below are given results and discussion for each of the experiments.

\subsection{Strong Scaling Study (speed up)}

\subsubsection{OpenMP}

The results for the speed up study for the OpenMP algorithm is given in
Fig.~\ref{ompsu}. As demonstrated, the algorithm does not demonstrate ideal speedup
(as expected), but is able to solve the
problem increasingly faster as the number of threads used is increased.  Some rough timing estimates indicate
that the $O(p)$ calculation in the OpenMP algorithm is $<0.1$ \% of the total runtime
(for 10$^9$ integers), which was on the order of 5 seconds.  Thus the limiting
factor is likely the overhead of managing threads and memory access times.  

The algorithm described by Alg.~\ref{alg:openmp} had to be modified slightly to achieve
the shown results (and all later OpenMP results).  It was modified to limit the cost
of memory access times.  In the corrected algorithm, each
thread was allowed to create its own memory within the
\verb{omp parallel{ section (essentially emulating distributed memory), for storing the input and prefix sums. This was
necessary to allow the program to correctly assign memory near the location of
the cores.  This significantly improved
the speedup, in particular going from 4 to 8 threads, which previously showed a drop
in speedup.  This drop was the result of processors having to go off chip to access their
portion of the array in RAM.  

\subsubsection{MPI}

The speedup results for the MPI algorithm are given in Fig.~\ref{mpisu}. A more
legible plot for the cases of $p\leq16$ is given in Fig.~\ref{mpizoomsu}.  The MPI
algorithm demonstrated slightly better speed up at equivalent core counts than the OpenMP
algorithm.  It is noted there is an irregular drop from 4 to 8 cores in the MPI speed
up plot.  This is likely due to the fact that 4 cores will be on 1 chip, whereas for 8
cores some of the RAM will be off chip, as discussed in the previous section and
machine specifications, leading to a decrease in expected efficiency gain.

The program continued to scale out to 128 and 256 cores, although at a
reduced rate of increase at higher core counts.  The rate of increase drops off at higher core counts due to the
increased cost of the communication in the $O(\log p)$ communication step, relative
to the $O(n/p)$ prefix sum steps by each processor.  There was much greater variance at 128 and 256 cores, due to much more
off node communication time, leading to some difficulty in determining how well the
speedup is increased at the high core counts. 

It is noted that although the MPI algorithm has the same theoretical time complexity as a linear array approach, there
is less communication steps ($O(\log p)$ versus $O(p)$ communications), which leads to overall a significantly more efficient algorithm as
the cost of the communication steps becomes the limiting factor at higher
MPI rank counts.  A linear array approach was also tested and found to not scale
past 64 cores due to the increased cost of communication.  This demonstrates an
inaccuracy in our model, which is expected because it doesn't account for
communication overhead.  Alg.~\ref{alg:mpi} has similar communication cost to a tree-traversal
based algorithm.

     \begin{figure}[htbp!]
         \centering
           \includegraphics[width=0.75\textwidth]{openmp2_speedup.pdf}
           \caption{Plot of speedup versus $p$ for OpenMP algorithm, for prefix sum
               of $10^9$ integers.\label{ompsu}}
     \end{figure}
\begin{figure}[htbp!]
           \centering
           \includegraphics[width=0.75\textwidth]{mpi_speedup.pdf}
           \caption{Plot of speedup versus $p$ for MPI algorithm, for prefix sum of
               $10^9$ integers.\label{mpisu}}
     \end{figure}
\begin{figure}[htbp!]
           \centering
           \includegraphics[width=0.75\textwidth]{small.pdf}
           \caption{Zoomed in plot of speedup versus $p$ for MPI
           algorithm, for prefix sum of 10$^9$ integers.\label{mpizoomsu}}
     \end{figure}

\clearpage

\subsection{Weak Scaling Study}

A plot of the weak scaling efficiency for various thread counts for the OpenMP and
MPI algorithms are given below.  In general, the weak scaling does not perform well for either
algorithm.  From 2-8 cores, the weak scaling efficiency are very similar for both the
MPI and OpenMP algorithm.  As noticed, the efficiency drops significantly from 2 to 8 cores.   The large initial
drops in the efficiency at low core counts are likely due to memory access times, as
there is minimal communication cost at these low core counts.   If the cause of the initial drop in efficiency at low core counts can be mitigated, it is likely the case that the total
efficiency would scale much better because this issue likely is effecting the
processors on all nodes locally.

For the MPI algorithm, the efficiency begins to level off at around 20\% above 16 cores.  It is noted however,
that above $\sim$32 cores, increasing the problem size, per core, does not result
in any loss in efficiency.  Although the MPI algorithm is not exceptionally
efficient, very large problems can be ran without much loss in computational time due
to additional parallel communication costs.

     \begin{figure}[htbp!]
         \centering
           \includegraphics[width=0.75\textwidth]{openmp_weak.pdf}
           \caption{Plot of weak scaling efficiency versus $p$ for OpenMP algorithm,
               for $10^8$ integers per processor.\label{ompw}}
     \end{figure}
\begin{figure}[htbp!]
           \centering
           \includegraphics[width=0.75\textwidth]{mpi_weak.pdf}
           \caption{Plot of weak scaling efficiency versus $p$ for MPI algorithm, for $10^8$ integers
               per processor.\label{mpiw}}
     \end{figure}
\begin{figure}[htbp!]
           \centering
           \includegraphics[width=0.75\textwidth]{mpi_weak_small.pdf}
           \caption{Zoomed in plot of weak scaling for MPI
           algorithm, for $10^8$ integers per processor.\label{mpizoomw}}
     \end{figure}
\clearpage
\subsection{Computational time versus problem size}

Plots of computational time versus problem size, for various \emph{fixed} $p$, are
given below for each algorithm.  It is expected that the dominant term in the
complexities is $O(n/p)$, so the computational times should scale linearly with $n$.
This behavior was observed, demonstrated by the linear shape of the plotted times.
For the case of $p=4$, for MPI, there is a slight
increase at 2$\times 10^9$. This increase is the result of the problem size being at the edge of the limits
of available RAM, likely resulting in an increased time due to many off chip memory
accesses.   

As $p$ is increased, the slope of the lines decreases because the time to solve the
same size of problems with more processors should be less, as expected.
The sequential times scales as $O(n)$, and the parallel algorithms should scale as
the dominate term of $O(n/p)$, where $p$ is fixed.  Thus, the ratio of the slopes of the
serial to the parallel line should be roughly equal to $1/p$.  By visual examination, this was found to be the
case (at least generally), indicating that in fact the $O(n/p)$ term is mostly
dominant through the regime of problems tested. 

     \begin{figure}[htbp!]
         \begin{subfigure}{0.5\textwidth}
         \centering
           \includegraphics[width=1.075\textwidth]{openmp_elems_4.pdf}
           \caption{$p=4$ \label{ompn4}}
     \end{subfigure}
     \begin{subfigure}{0.5\textwidth}
           \centering
           \includegraphics[width=1.075\textwidth]{openmp_elems_8.pdf}
           \caption{$p=8$ \label{ompn8}}
       \end{subfigure}
        \caption{Plot of time $T_{exp}$ vs problem size $n$ for OpenMP
           algorithm.\label{ompn}}
     \end{figure}
     \begin{figure}[htbp!]
         \begin{subfigure}{0.5\textwidth}
         \centering
           \includegraphics[width=1.075\textwidth]{mpi_elems_4.pdf}
           \caption{$p=4$ \label{mpin4}}
     \end{subfigure}
     \begin{subfigure}{0.5\textwidth}
           \centering
           \includegraphics[width=1.075\textwidth]{mpi_elems_16.pdf}
           \caption{$p=16$ \label{mpin16}}
       \end{subfigure}
        \caption{Plot of time $T_{exp}$ vs problem size $n$ for MPI
           algorithm.\label{mpin}}
     \end{figure}



\subsection{Determining Asymptotic Coefficients}

\subsubsection{OpenMP}

The experiment to determine the asymptotic coefficients as discussing in the
experimental set up was performed for different processor counts for both algorithms.
Assuming the $O(n/p)$ term dominates, the plotted coefficients represent an
approximate experimental time of the form
\begin{equation}
    T_{exp}(n,p) \simeq C_{0}\frac{n}{p}, \quad n>n_0
    \label{duh}
\end{equation}
A table summarizing the visually estimated coefficients, for both algorithms and various $p$,  is
given in Table~\ref{tab1}. Figures are given below, with zoomed in photos given with
the removed data points to resolve the data points. The fact that $C_0$ is increasing
with $p$, indicates that there is extra terms the model in Eq.~\eqref{duh} is not
accounting for.  This is clearly true as we have not attempted to determine the
coefficients in front of the parallel step $O(p)$ or $O(\log p)$.  The
variablity in the asymptotic coefficient is partially due to the fact that the
communication time is more significant, relative tot he total run time, as compared
to the total run time.  Since more processors are used by the case of $p=64$, the
$n/p$ term becomes not the dominant term in the model.

\begin{table}[h!]
    \caption{\label{tab1} Tabulated results for estimating asymptotic coefficients}
\centering
\begin{tabular}{|c|c|c|} \hline
    $p$ & $C_0$ & $n_0$ \\ \hline
    \multicolumn{3}{|c|}{OpenMP} \\ \hline
     4 & 0.004 & $0.1\times10^9$ \\
     8 & 0.0075 & $0.1\times10^9$ \\ \hline
    \multicolumn{3}{|c|}{MPI} \\ \hline
     4 & 0.0044 & $0.4\times10^9$ \\
     16 & 0.0095 & $0.4\times10^9$ \\
     64 & 0.0150 & $1.0\times10^9$ \\ \hline
\end{tabular}
\end{table}

     \begin{figure}[htbp!]
         \begin{subfigure}{0.5\textwidth}
         \centering
           \includegraphics[width=1.075\textwidth]{openmp_asym_4.pdf}
           \caption{$p=4$ \label{ompn4}}
     \end{subfigure}
     \begin{subfigure}{0.5\textwidth}
           \centering
           \includegraphics[width=1.075\textwidth]{openmp_asym_8.pdf}
           \caption{$p=8$ \label{ompn8}}
       \end{subfigure}
     \begin{subfigure}{0.5\textwidth}
           \centering
           \includegraphics[width=1.075\textwidth]{openmp_asym_zoom_8.pdf}
           \caption{$p=8$, enlarged \label{ompnzoom8}}
       \end{subfigure}
       \caption{Plot of $T_{exp}/T_{pred}$ vs problem size $n$ for OpenMP
           algorithm.\label{ompn}}
     \end{figure}

\subsubsection{MPI}


     \begin{figure}[htbp!]
         \begin{subfigure}{0.5\textwidth}
         \centering
           \includegraphics[width=1.075\textwidth]{mpi_asym_4.pdf}
           \caption{$p=4$ \label{mpin4}}
     \end{subfigure}
     \begin{subfigure}{0.5\textwidth}
           \centering
           \includegraphics[width=1.075\textwidth]{mpi_asym_16.pdf}
           \caption{$p=16$ \label{mpin16}}
       \end{subfigure}
     \begin{subfigure}{0.5\textwidth}
           \centering
           \includegraphics[width=1.075\textwidth]{mpi_asym_zoom_16.pdf}
           \caption{$p=16$, enlarged \label{mpinzoom16}}
       \end{subfigure}
     \begin{subfigure}{0.5\textwidth}
           \centering
           \includegraphics[width=1.075\textwidth]{mpi_asym_64.pdf}
           \caption{$p=64$ \label{mpin64}}
       \end{subfigure}
     \begin{subfigure}{0.5\textwidth}
           \centering
           \includegraphics[width=1.075\textwidth]{mpi_asym_zoom_64.pdf}
           \caption{$p=64$, enlarged \label{mpinzoom64}}
       \end{subfigure}
       \caption{Plot of $T_{exp}/T_{pred}$ vs problem size $n$ for MPI
           algorithm.\label{mpin}}
     \end{figure}

\section{Conclusions}

Since the prefix sum involves such primitive operations, it exposed any memory or
other overhead in computations.  It took very efficient code to demonstrate much speed
up. The weak scaling experiment demonstrated that there is extra costs that we are
not expecting.

\end{document}



